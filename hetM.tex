%\documentclass{article}
\documentclass[12pt]{article}
\usepackage{latexsym}
\usepackage{amsmath}
\usepackage{amssymb}
\usepackage{relsize}
\usepackage{geometry}
\geometry{letterpaper}

\usepackage{showlabels}

\textwidth = 6.0 in
\textheight = 8.5 in
\oddsidemargin = 0.0 in
\evensidemargin = 0.0 in
\topmargin = 0.2 in
\headheight = 0.0 in
\headsep = 0.0 in
%\parskip = 0.05in
\parindent = 0.35in


%% common definitions
\def\stackunder#1#2{\mathrel{\mathop{#2}\limits_{#1}}}
\def\beqn{\begin{eqnarray}}
\def\eeqn{\end{eqnarray}}
\def\nn{\nonumber}
\def\baselinestretch{1.1}
\def\beq{\begin{equation}}
\def\eeq{\end{equation}}
\def\ba{\beq\new\begin{array}{c}}
\def\ea{\end{array}\eeq}
\def\be{\ba}
\def\ee{\ea}
\def\stackreb#1#2{\mathrel{\mathop{#2}\limits_{#1}}}
\def\Tr{{\rm Tr}}
\newcommand{\gsim}{\lower.7ex\hbox{$
\;\stackrel{\textstyle>}{\sim}\;$}}
\newcommand{\lsim}{\lower.7ex\hbox{$
\;\stackrel{\textstyle<}{\sim}\;$}}
\newcommand{\nfour}{${\mathcal N}=4$ }
\newcommand{\ntwo}{${\mathcal N}=2$ }
\newcommand{\ntwon}{${\mathcal N}=2$}
\newcommand{\ntwot}{${\mathcal N}= \left(2,2\right) $ }
\newcommand{\ntwoo}{${\mathcal N}= \left(0,2\right) $ }
\newcommand{\none}{${\mathcal N}=1$ }
\newcommand{\nonen}{${\mathcal N}=1$}
\newcommand{\vp}{\varphi}
\newcommand{\pt}{\partial}
\newcommand{\ve}{\varepsilon}
\newcommand{\gs}{g^{2}}
\newcommand{\qt}{\tilde q}
\renewcommand{\theequation}{\thesection.\arabic{equation}}

%%
\newcommand{\p}{\partial}
\newcommand{\wt}{\widetilde}
\newcommand{\ov}{\overline}
\newcommand{\mc}[1]{\mathcal{#1}}
\newcommand{\md}{\mathcal{D}}

\newcommand{\GeV}{{\rm GeV}}
\newcommand{\eV}{{\rm eV}}
\newcommand{\Heff}{{\mathcal{H}_{\rm eff}}}
\newcommand{\Leff}{{\mathcal{L}_{\rm eff}}}
\newcommand{\el}{{\rm EM}}
\newcommand{\uflavor}{\mathbf{1}_{\rm flavor}}
\newcommand{\lgr}{\left\lgroup}
\newcommand{\rgr}{\right\rgroup}

\newcommand{\Mpl}{M_{\rm Pl}}
\newcommand{\suc}{{{\rm SU}_{\rm C}(3)}}
\newcommand{\sul}{{{\rm SU}_{\rm L}(2)}}
\newcommand{\sutw}{{\rm SU}(2)}
\newcommand{\suth}{{\rm SU}(3)}
\newcommand{\ue}{{\rm U}(1)}
%%%%%%%%%%%%%%%%%%%%%%%%%%%%%%%%%%%%%%%
%  Slash character...
\def\slashed#1{\setbox0=\hbox{$#1$}             % set a box for #1
   \dimen0=\wd0                                 % and get its size
   \setbox1=\hbox{/} \dimen1=\wd1               % get size of /
   \ifdim\dimen0>\dimen1                        % #1 is bigger
      \rlap{\hbox to \dimen0{\hfil/\hfil}}      % so center / in box
      #1                                        % and print #1
   \else                                        % / is bigger
      \rlap{\hbox to \dimen1{\hfil$#1$\hfil}}   % so center #1
      /                                         % and print /
   \fi}                                        %

%%EXAMPLE:  $\slashed{E}$ or $\slashed{E}_{t}$

%%

\newcommand{\LN}{\Lambda_\text{SU($N$)}}
\newcommand{\sunu}{{\rm SU($N$) $\times$ U(1) }}
\newcommand{\sunun}{{\rm SU($N$) $\times$ U(1)}}
\def\cfl {$\text{SU($N$)}_{\rm C+F}$ }
\def\cfln {$\text{SU($N$)}_{\rm C+F}$}
\newcommand{\mUp}{m_{\rm U(1)}^{+}}
\newcommand{\mUm}{m_{\rm U(1)}^{-}}
\newcommand{\mNp}{m_\text{SU($N$)}^{+}}
\newcommand{\mNm}{m_\text{SU($N$)}^{-}}
\newcommand{\AU}{\mc{A}^{\rm U(1)}}
\newcommand{\AN}{\mc{A}^\text{SU($N$)}}
\newcommand{\aU}{a^{\rm U(1)}}
\newcommand{\aN}{a^\text{SU($N$)}}
\newcommand{\baU}{\ov{a}{}^{\rm U(1)}}
\newcommand{\baN}{\ov{a}{}^\text{SU($N$)}}
\newcommand{\lU}{\lambda^{\rm U(1)}}
\newcommand{\lN}{\lambda^\text{SU($N$)}}
%\newcommand{\Tr}{{\rm Tr\,}}
\newcommand{\bxir}{\ov{\xi}{}_R}
\newcommand{\bxil}{\ov{\xi}{}_L}
\newcommand{\xir}{\xi_R}
\newcommand{\xil}{\xi_L}
\newcommand{\bzl}{\ov{\zeta}{}_L}
\newcommand{\bzr}{\ov{\zeta}{}_R}
\newcommand{\zr}{\zeta_R}
\newcommand{\zl}{\zeta_L}
\newcommand{\nbar}{\ov{n}}

\newcommand{\pts}{\psi_{\rm tr}^0}
\newcommand{\ptN}{\psi_{\rm tr}^N}
\newcommand{\cts}{\chi_{\rm tr}^0}
\newcommand{\ctN}{\chi_{\rm tr}^N}
\newcommand{\tts}{\vartheta_{\rm tr}^0}
\newcommand{\ttN}{\vartheta_{\rm tr}^N}
\newcommand{\rts}{\rho_{\rm tr}^0}
\newcommand{\rtN}{\rho_{\rm tr}^N}

\newcommand{\tor}{\vartheta_{\rm or}}
\newcommand{\eor}{\eta_{\rm or}}
\newcommand{\kor}{\kappa_{\rm or}}
\newcommand{\sor}{\sigma_{\rm or}}
\newcommand{\por}{\psi_{\rm or}}
\newcommand{\cor}{\chi_{\rm or}}
\newcommand{\uor}{\upsilon_{\rm or}}
\newcommand{\oor}{\omega_{\rm or}}

\newcommand{\CPC}{CP($N-1$)$\times$C }
\newcommand{\CPCn}{CP($N-1$)$\times$C}

\newcommand{\MN}{M_\text{SU($N$)}}
\newcommand{\MU}{M_{\rm U(1)}}

\begin{document}



\begin{titlepage}

\begin{flushright}
FTPI-MINN-09/10 UMN-TH-2740/09\\
%ITEP-TH-XX/09\\
February 24, 2009
\end{flushright}

\begin{center}

{\Large \bf   Description of the Heterotic String Solutions
\\[2mm]
 in the \boldmath{$M$} Model }
\end{center}

\begin{center}
{\bf P. A. Bolokhov$^{a,b}$, M.~Shifman$^{c}$, and \bf A.~Yung$^{c,d}$}
\end {center}
\vspace{0.3cm}
\begin{center}

$^a${\it Physics and Astronomy Department, University of Pittsburgh, Pittsburgh, Pennsylvania, 15260, USA}\\
$^b${\it Theoretical Physics Department, St.Petersburg State University, Ulyanovskaya~1, 
	 Peterhof, St.Petersburg, 198504, Russia}\\
$^c${\it  William I. Fine Theoretical Physics Institute,
University of Minnesota,
Minneapolis, MN 55455, USA}\\
$^{d}${\it Petersburg Nuclear Physics Institute, Gatchina, St. Petersburg
188300, Russia}\\
%$^e${\it Institute of Theoretical and Experimental Physics, Moscow
%117259, Russia}
\end{center}

\begin{abstract}

We continue the  study of heterotic non-Abelian BPS-saturated flux tubes (strings).
Previously, such solutions were obtained  in  U($N$) gauge theories:
 \ntwo super\-symmetric QCD deformed by
superpotential terms $\mu{\mathcal A}^2$ breaking
\ntwo supersymmetry down to \none$\!\!$. 
In these models one cannot consider the limit $\mu\to\infty$
which would eliminate adjoint fields: the bulk theory develops a Higgs
branch; the emergence of massless particles in the bulk
precludes one from taking the  limit $\mu\to\infty$.
This drawback is absent in the $M$ model (hep-th/0701040)
where the matter sector includes additional ``meson"
fields $M$ introduced in a special way.
We generalize our previous results 
to the $M$ model, derive the
 heterotic string (the 
 string world-sheet theory is
 a heterotic \ntwoo 
sigma model, with the CP$(N-1)$ target space for bosonic fields and an extra 
right-handed fermion coupled to the fermion fields of the
\ntwot CP$(N-1)$ model),
 and then explicitly obtain all relevant zero modes.
This allows us to relate parameters of the
microscopic $M$ model to those of  the world-sheet theory. 
% This relation is qualitatively similar to our previous results, except that the role of $\mu$ is now taken
%by $h$, the coupling constant of the $M$ fields. 
The limit $\mu\to\infty$ is perfectly smooth.
Thus, the full-blown and fully analyzed heterotic string emerges, for the first time, in
the \none theory with no adjoint fields. The fate of the confined monopoles is discussed.

 

\end{abstract}

\end{titlepage}

\section{Introduction}

Non-Abelian BPS-saturated flux tubes were discovered \cite{HT1,ABEKY} and studied 
\cite{SYmon,Tong,HT2} in \ntwo super\-symmetric QCD with the gauge group
U$(N)$, the Fayet--Iliopoulos (FI) term,
and $N$ flavors ($N$ hypermultiplets in the fundamental representation),
for a review see \cite{Trev,SYrev,Jrev,Trev2}.
If \ntwo supersymmetry is maintained in the bulk,
the low-energy theory on the
string world sheet is split into two disconnected parts:
a free theory for (super)translational moduli and a nontrivial part, a theory of
interacting (super)orientational moduli described by \ntwo supersymmeytic CP$(N-1)$ model.
 The above splitting of the moduli space
 is completely fixed by the fact
that the basic bulk theory has eight supercharges, and the string under consideration
is 1/2 BPS (classically). 

If \ntwo bulk theory is deformed by mass terms $\mu{\mathcal A}^2$ of the adjoint fields
breaking \ntwo down to \nonen, the situation drastically changes:
two of the four former supertranslational modes
become coupled to two superorientational modes \cite{Edalati}.
As a result, the world sheet theory is deformed too.
Instead of the well-studied ${\mathcal N}=(2,2)$ CP$(N-1)$ model we now have
a heterotic \ntwoo 
sigma model, with the CP$(N-1)$ target space for bosonic fields and an extra 
right-handed fermion which couples to the fermion fields of the
CP$(N-1)$ model in a special way. In the previous works \cite{SYhet,BSYhet}
the heterotic world-sheet model was derived from the microscopic theory by
a direct calculation of all relevant zero modes.
This allowed us 
to relate the  heterotic \ntwoo 
sigma model parameters with those of the bulk theory.

The task we addressed was moving away from \ntwo$\!\!$, towards \none$\!\!$.
In particular, it is highly desirable to get rid of all adjoint fields
inherent to \none models.  If we were able to tend $\mu\to\infty$ this goal would be achieved,
all adjoint fields would become infinitely heavy and could be eliminated. Unfortunately,
simultaneously with increasing the masses of the adjoint fields the bulk theory develops 
a Higgs branch, and massless (light) moduli fields coming with it. The strings swells,
and all approximations fail at $\mu >\mu_*$ where $\mu_*$ is a critical value,
\beq
\mu_* \sim \frac{\xi}{\Lambda_\sigma}\,,
\label{critmu}
\eeq
$\xi$ is the FI coefficient and $\Lambda_\sigma$ is the dynamical scale
of the world-sheet sigma model.
Although $\mu_*$ can be made large, there are crucial questions which cannot be addressed
under the constraint $\mu \ll \mu_*$. One of them is the fate of the kinks in the heterotic
CP$(N-1)$ model which, from the bulk standpoint, represent confined monopoles.
At $\mu \ll \mu_*$ the world-sheet theory has $N$ degenerate (albeit quantum-mechanically nonsupersymmetric) vacua
which are well defined. Correspondingly, the kink masses are well-defined too;
in fact, they were calculated \cite{SYhet2} in the large-$N$ limit. In a formal limit $\mu\to\infty$
the above degenerate vacua coalesce. Physics of the kinks becomes obscure.

To avoid this problem an $M$ model was designed \cite{GSYmmodel}. Besides the fields present in the
\none deformation of the basic \ntwo bulk theory, the $M$ model includes $N^2$
``mesonic" superfields, which break \ntwo right from the start. The $M$ model is characterized by one
extra interaction constant $h$. 
It was demonstrated \cite{GSYmmodel} that at finite $h$ the limit $\mu\to\infty$ becomes smooth.
Therefore, one can completely eliminate the adjoint fields.
The solitonic flux tube solution (BPS saturated at the classical level) persists.  
Our task in this paper is to derive the world-sheet theory for these strings (in the limit of
small $h$ and $\mu\to\infty$). We prove that it is the same heterotic
CP$(N-1)$ model, with specific relations between constants of this model and those of the bulk theory.
To obtain these relations we determine all relevant zero modes for the $M$-model flux-tube solution.

%In brief, our findings are as follows.




The paper is organized as follows. In Sect.~2 we review $M$ model, string solutions and 
the bosonic part of the world sheet theory on the non-Abelian string. In Sect.~3 we calculate
fermionic supertranslational and superorientational modes of the string and derive
the fermionic part of the world sheet theory. Our derivation shows that the world sheet theory is
the  heterotic \ntwoo supersymmetric CP$(N-1)$ model. In Sect.~4 we discuss the fate of bulk monopoles
confined on the string in the limit of large $\mu$. Sect.~5 contains our brief conclusions.
Our notations are summarized  in Appendix of Ref.~\cite{BSYhet}.


%%%%%%%%%%%%%%%%%%%%%%%%%%%%
%\tableofcontents
%%%%%%%%%%%%%%%%%%%%%%%%%%%%



%%%%%%%%%%%%%%%%%%%%%%%%%%%%%%%%%%%%%%%%%%%%%%%%%%%%%%%%%%%%%%%%%%%%%%%%%%%%%%%%%%
%
%	  		        S E C T I O N
%
%%%%%%%%%%%%%%%%%%%%%%%%%%%%%%%%%%%%%%%%%%%%%%%%%%%%%%%%%%%%%%%%%%%%%%%%%%%%%%%%%%
%\section{Introduction}


%%%%%%%%%%%%%%%%%%%%%%%%%%%%%%%%%%%%%%%%%%%%%%%%%%%%%%%%%%%%%%%%%%%%%%%%%%%%%%%%%%
%
%	  		        S E C T I O N
%
%%%%%%%%%%%%%%%%%%%%%%%%%%%%%%%%%%%%%%%%%%%%%%%%%%%%%%%%%%%%%%%%%%%%%%%%%%%%%%%%%%
\section{{\boldmath $M$}-Model}
\setcounter{equation}{0}

In this section we describe how non-Abelian strings emerge in the $M$ Model.
Our discussion here parallels that of \cite{GSYmmodel} where more details are
given, and we review only the most essential points.
The theory we start with is the \ntwo \sunu SQCD with the \ntwo supersymmetry
broken to \none by the following deformations
\beq
\label{none}
	\delta\mc{W}_{\mc{N}=1} ~~=~~ \sqrt{2N}\,\mu_1\, \left( \mc{A}^{\rm U(1)} \right)^2 
				~+~ \frac{\mu_2}{2}\, (\mc{A}^a)^2
				~+~ \Tr\, M\, \wt{Q}\, Q\,.
\eeq
Here the first two terms break supersymmetry by giving masses to the adjoint 
supermultiplets $ \mc{A}^{\rm U(1)} $ and $ \mc{A}^\text{SU($N$)} $, 
while $ M $ breaks supersymmetry by coupling to the quark fields.
$ M $ is the superfield extension of the quark mass $ m_A^B $, and is by itself a dynamical field,
\[
	\delta S_{M \rm kin} ~~=~~ \int\, d^4x\, d^2\theta\, d^2\ov{\theta} \;
					\frac{2}{h}\, \Tr\, \ov{M}\, M \,,
\]
where $ h $ is a dimensionless coupling constant. 

The purpose of introduction of the masses $ \mu_1 $ and $ \mu_2 $ is to make the adjoint fields
heavy and exclude them from low-energy physics.
The role of the superfield $ M $ is to lift the Higgs branch which appears when
the adjoints are integrated out. If coupling $h=0$, the $M$-field is frozen to a constant 
and plays a role of the quark mass. This does not break \ntwo supersymmetry. However,
once $h\neq 0$ the coupling to $M$ becomes a deformation, which breaks \ntwo supersymmetry
down to \none (together with parameters $\mu_1$ and $\mu_2$).

The bosonic part of the theory is
\begin{align}
%
\label{theory}
	S_{\rm bos} ~~=~~ & \int d^4 x 
		\lgr
			\frac{1}{2g_2^2}\Tr \left(F_{\mu\nu}^\text{SU($N$)}\right)^2  ~+~
			\frac{1}{g_1^2} \left(F_{\mu\nu}^{\rm U(1)}\right)^2 ~+~ 
			\right. 
			\\[2mm]
%
\notag
		&
			\phantom{\int d^4 x \lgr\right.\,}
			\frac{2}{g_2^2}\Tr \left|\nabla_\mu \aN \right|^2   ~+~
			\frac{4}{g_1^2} \left|\p_\mu \aU \right|^2
			~+~
			\left| \nabla_\mu q^A \right|^2 ~+~ \left|\nabla_\mu \ov{\wt{q}}{}^A \right|^2 
			~+~
			\\[2mm]
%
\notag
		&
			\phantom{\int d^4 x \lgr\right.\,}
		\left.
			\frac{1}{h}\, \left| \p_\mu M^0 \right|^2  ~+~
			\frac{1}{h}\, \left| \p_\mu M^a \right|^2 ~+~
			V(q^A, \wt{q}{}_A, a^a, \aU, M^0, M^a)
		\rgr .
\end{align}

	Here $ \nabla_\mu $ is the covariant derivative in the appropriate representation
\begin{align*}
%
	\nabla_\mu^{\rm adj} & ~~=~~ \p_\mu  ~-~ i\, [ A_\mu^a T^a, \;\cdot\;]~, \\[3mm]
%
	\nabla_\mu^{\rm fund} & ~~=~~ \p_\mu ~-~ i\,A^{\rm U(1)}_\mu ~-~ i\, A_\mu^a T^a\,.
\end{align*}
Vector fields $A_{\mu}$ and complex scalars $a$ belong to gauge multiplets of U(1) and SU$(N)$ sectors, while $q^{kA}$, $\tilde{q}_{Ak}$ denote squarks, $k=1,...,N$ and $A=1,...,N_f$ are
color and flavor indices respectively. In this paper we consider the case $N_f=N$.

	The matrix superfield $ M^A_B $ is decomposed as 
\[
	M^A_B ~~=~~ \frac{1}{2}\,\delta^A_B\, M^0  ~+~ (T^a)^A_B\, M^a\,.
\]

	The potential of the theory \eqref{theory} takes the form
\begin{align}
%
\notag
	& V(q^A, \wt{q}{}_A, a^a, \aU, M^0, M^a) ~~=~~ 
	\\[3mm]
%
\notag
	&\qquad\quad ~~=~~
			\frac{g_2^2}{2} \left( \frac{1}{g_2^2}\,f^{abc}\ov{a}{}^b a^c 
				~+~ \Tr\, \ov{q}\, T^a q 
				~-~ \Tr\, \wt{q}\, T^a \ov{\wt{q}} \right)^2 
	\\[3mm]
%
\label{V}
	&\qquad\quad ~~+~~
		\frac{g_1^2}{8}\, (\Tr\, \ov{q} q ~-~ \Tr\, \wt{q} \ov{\wt{q}} ~-~ N\xi )^2
	\\[3mm]
%
\notag
	&\qquad\quad ~~+~~
		2g_2^2\, \Bigl| \Tr\, \wt{q}\,T^a q ~+~ 
			\frac{1}{\sqrt{2}}\, \frac{\p\mc{W}_{\mc{N}=1}}{\p a^a} \Bigr|^2
	~+~
	\frac{g_1^2}{2}\, \Bigl| \Tr\, \wt{q} q ~+~ 
			\frac{1}{\sqrt{2}}\, \frac{\p\mc{W}_{\mc{N}=1}}{\p\aU} \Bigr|^2
	\\[3mm]
%
\notag
	&\qquad\quad ~~+~~
	2\, \Tr \Biggl\{  
		\left| \left( \aU ~+~ a^a T^a \right) q ~+~
			\frac{1}{\sqrt{2}}\, q\left( \frac{M^0}{2} \;+\; M^a\,T^a \right) \right|^2 
	\\[3mm]
%
\notag
	&\phantom{\qquad\quad ~~+~~ 2 } ~+~
		\left| \left( \aU ~+~ a^a T^a \right) \ov{\wt{q}} ~+~
			\frac{1}{\sqrt{2}}\, \ov{\wt{q}}\left( \frac{M^0}{2} \;+\; M^a\,T^a \right) \right|^2 
			\Biggr\}
	\\[3mm]
%
\notag
	&\qquad\quad ~~+~~
	\frac{h}{4}\, \left| \Tr\, \wt{q}q \right|^2  ~+~ h\,\left| \Tr\, q\,T^a\,\wt{q} \,\right|^2	
	~.
\end{align}
	This potential is a sum of $ F $ and $ D $ terms, in particular the last two terms in \eqref{V}
	are the $ F $ terms of the  $ M $ field.
	On the third line we have also introduced a Fayet--Iliopoulos $ D $-term, with the real
	parameter $ \xi $.
	The \ntwo supersymmetry is broken by parameters $ \mu_1 $, $ \mu_2 $ and $ h $ via $ \mc{W}_{\mc{N}=1} $, while
	the FI-term does not break \ntwo supersymmetry \cite{HSZ,VY}.

	A non-zero $ \xi $ in the potential triggers condensation of quarks and spontaneous breaking 
	of the gauge symmetry. 
	The vacuum expectation values (VEVs) of the quarks can be chosen in the form
\begin{align}
%
\notag
&
	\langle q^{kA} \rangle ~=~ \sqrt{\xi} 
		\begin{pmatrix}
			 1  &   0  &  ... \\
			... &  ... &  ... \\
			... &   0  &  1 
		\end{pmatrix} ~,
	\qquad\qquad 
	\langle \ov{\wt{q}}{}^{kA} \rangle ~=~ 0~,
	\\
%
\label{qvev}
&
	\qquad\qquad  k~=~ 1,...\, N~, \qquad  A ~=~ 1,...\, N~,
\end{align}
	the so-called colour-flavor locked form.
	The adjoint VEVs have to vanish classically,
\beq
\label{avev}
	\langle \aN \rangle  ~~=~~ 0~, \qquad\qquad  \langle \aU \rangle ~~=~~ 0~,
\eeq
	together with the VEVs of the $ M $-field,
\beq
\label{Mvev}
	\langle M^a \rangle ~~=~~ 0~, \qquad\qquad  \langle M^0 \rangle ~~=~~ 0~.
\eeq

	Despite the full higgsing of the gauge symmetry, the VEVs \eqref{qvev},
	\eqref{avev}, and \eqref{Mvev} leave a global diagonal \cfl symmetry unbroken
\beq
\label{c+f}
	q ~~\to~~ UqU^{-1}\,, \qquad \aN ~~\to~~ U \aN U^{-1}\,, \qquad
		M ~~\to~~ UMU^{-1}\,.
\eeq
	In what follows, we will be interested in the limit of very large $ \mu_1 $, $ \mu_2 $.
	It appears that the VEV structure \eqref{qvev}, \eqref{avev} and \eqref{Mvev} does not
	depend on the supersymmetry breaking parameters, owing to the fact that 
	the adjoint fields vanish in the vacuum, Eq.~\eqref{avev}.
	In particular, the VEVs will retain the same form up to very large $ \mu $. 

	To allow the theory to be treated semiclassically, we arrange it to be at weak
	coupling, by separating the dynamical scale of SU($N$) from the scale of the gauge
	symmetry breaking $ \xi $:
\[
 \sqrt{\xi} ~\gg~ \LN \,.
\]

	The perturbative spectrum was discussed in detail in \cite{GSYmmodel}, and we will only
	concentrate on the limit of large $ \mu $. 
	Regardless of $ \mu $, the gauge bosons acquire mass,
\beq
\label{phmass}
	m_{\rm ph} ~~=~~ g_1\, \sqrt{\frac{N}{2}\,\xi}
\eeq
	for the U(1) gauge boson (``photon'') and
\beq
\label{wmass}
	m_W ~~=~~ g_2\, \sqrt{\xi}
\eeq
	for the SU($N$) bosons.

	The scalar bosons line up in the following hierarchy of scales.
	The heaviest bosons, in the $ \mu_i \gg \sqrt{\xi} $ limit, have the masses
\begin{align}
\label{amass}
%
\notag
	m_{\rm U(1)}^{\rm(largest)} & ~~=~~ \sqrt{\frac{N}{2}}\, g_1^2\mu_1 \,, \\
%
	m_\text{SU($N$)}^{\rm(largest)} & ~~=~~ \phantom{\sqrt{\frac{N}{2}}\, }
						g_2^2\mu_2\,,
\end{align}
	with the first mass carried by two degenerate states, while
	the second mass is carried by $ 2 (N^2 - 1) $ states.
	These are the masses of heavy adjoint scalars $ \aU $ and $ \aN $.

	The low-energy bulk theory spectrum consists of light states with the masses
\begin{align}
\label{U1mass}
%
\notag
	m_\text{U(1)}^{(1)}  ~~=~~ \sqrt{\frac{h N\xi}{4}}
		\left\{  1 ~+~ \frac{\sqrt{\xi}}{2g_1\mu_1}\,\sqrt{\gamma_1 (\gamma_1 + 1)} 
				~+~ \cdots \right\}\,,	
	\\
%
	m_\text{U(1)}^{(2)}  ~~=~~ \sqrt{\frac{h N\xi}{4}}
		\left\{  1 ~-~ \frac{\sqrt{\xi}}{2g_1\mu_1}\,\sqrt{\gamma_1 (\gamma_1 + 1)}
				~+~ \cdots \right\}\,,
\end{align}
	with two states for each, and of states with the masses
\begin{align}
\label{SUNmass}
%
\notag
	m_\text{SU($N$)}^{(1)} ~~=~~ \sqrt{\frac{h\xi}{2}}
		\left\{ 1 ~+~ \frac{\sqrt{\xi}}{2 g_2\mu_2}\,\sqrt{\gamma_2 (\gamma_2 + 1)}
				~+~ \cdots \right\}\,,
	\\
%
	m_\text{SU($N$)}^{(2)} ~~=~~ \sqrt{\frac{h\xi}{2}}
		\left\{ 1 ~-~ \frac{\sqrt{\xi}}{2 g_2\mu_2}\,\sqrt{\gamma_2 (\gamma_2 + 1)}
				~+~ \cdots \right\}\,,
\end{align}
	with $ 2 (N^2 - 1) $ degenerate states for each of the values.

	At non-zero $ h $ there are no massless states in the bulk theory, even
	if $ \mu_i \to \infty $.
	One can integrate out the heavy adjoint fields, obtaining
\begin{align}
\label{mmodel}
%
\notag
	S_{\rm bos}^\text{$M$-model} ~=~ & \int d^4 x 
		\biggl\lgroup
			\frac{1}{2g_2^2}\Tr \left(F_{\mu\nu}^\text{SU($N$)}\right)^2  ~+~
			\frac{1}{g_1^2} \left(F_{\mu\nu}^{\rm U(1)}\right)^2 ~+~ 
			\Tr\, \left| \nabla_\mu q \right|^2 ~+~ \Tr\,\left|\nabla_\mu \ov{\wt{q}}\, \right|^2 
			\\[2mm]
%
		&
\notag
			\phantom{\int d^4 x}
			~+~
			\frac{1}{h}\, \left| \p_\mu M^0 \right|^2  ~+~
			\frac{1}{h}\, \left| \p_\mu M^a \right|^2 ~+~
			\frac{g_2^2}{2} ( \Tr\, \ov{q}\, T^a q 
					~-~ \Tr\, \wt{q}\, T^a \ov{\wt{q}} )^2 
			\\[2mm]
%
\notag
		&
			\phantom{\int d^4 x}
			~+~
			\frac{g_1^2}{8}\, (\Tr\, \ov{q} q ~-~ \Tr\, \wt{q} \ov{\wt{q}} ~-~ N\xi )^2
			~+~
			\Tr\, |q\,M|^2 ~+~ \Tr\, |\ov{\wt{q}}\, M |^2
			\\[2mm]
%
		&
			\phantom{\int d^4 x}
			~+~
			\frac{h}{4}\, \left| \Tr\, \wt{q}q \right|^2  ~+~ 
			h\,\left| \Tr\, q\,T^a\,\wt{q} \,\right|^2
			\biggr\rgroup\,.	
\end{align}
	The vacuum of this theory is given by Eqs.~\eqref{qvev} and \eqref{Mvev}.
	The perturbative excitations consist of the \none gauge multiplets with masses
	\eqref{phmass} and \eqref{wmass}, and of chiral multiplets with masses determined by the 
	leading terms in Eqs.~\eqref{U1mass} and \eqref{SUNmass}.
	The scale of the \none theory is related to that of the original \ntwo theory as
\beq
	\Lambda_{\mc{N}=1}^{2N} ~~=~~ \mu_2^N\, \Lambda^N_\text{SU($N$)}\,.
\label{Lambda}
\eeq
	By taking the FI parameter large enough, 
	$ g_2\sqrt{\xi} \gg \Lambda_{\mc{N}=1} $
	we ensure the \none theory (the $M$-model) is at
	weak coupling.
	
	The theory \eqref{mmodel} admits the existence of non-Abelian strings, the presence of which can be traced
	from the \ntwo theory \eqref{theory}.
	A $ Z_N $ string can be written in terms of profile functions \cite{ABEKY,GSYmmodel}
\begin{align}
%
\notag
	q   ~~=~~  &
		\lgr \begin{matrix}
			\phi_2(r) & 0     & \dots      & 0      \\
			\dots     & \dots & \dots      & \dots  \\
			0         & \dots & \phi_2(r)  & 0      \\
			0         &  0    & \dots      & e^{i\alpha}\phi_1(r) 
		     \end{matrix} \rgr ,
		\qquad \wt{q} ~~=~~ 0\,,
	\\
%
\label{znstr}
	\\[-0.7cm]
%
\notag
	A_i^\text{SU($N$)}  ~~=~~
		\frac{1}{N} & \lgr \begin{matrix}
        			    	1       &   \dots   &  0       &   0   \\
        				\dots   &   \dots   &  \dots   & \dots \\
        				0       &   \dots   &  1       &   0   \\
        				0       &     0     &  \dots   & - (N-1) 
	   		         \end{matrix} \rgr
		(\p_i\alpha)\bigl( -1 ~+~ f_{N}(r) \bigr)\,,
	\\
%
\notag
	A_i^{\rm U(1)}  ~~=~~ & \frac{1}{N}(\p_i\alpha)\lgr 1 ~-~ f(r) \rgr \cdot \mathlarger{\mathbf{1}}\,,
	\qquad 
	A_0^{\rm U(1)} ~=~ A_0^\text{SU($N$)} ~=~ 0\,,
	\\
%
\notag
	\aU ~~=~~ & \aN ~~=~~ M^0 ~~=~~ M^a ~~=~~ 0\,,
\end{align}
	where $ r $ and $ \alpha $ are the polar coordinates in the plane orthogonal to the string, while 
	index $ i = 1,~ 2 $ labels the Descartes coordinates in this plane.
	The quark profile functions $ \phi_1(r) $, $ \phi_2(r) $, and the gauge profile functions 
	$ f(r) $ and $ f_N(r) $ obey a system of first-order differential equations
\begin{align}
%
\notag
&	\p_r\, \phi_1(r) ~-~ \frac{1}{Nr}\, \lgr f(r) ~+~ (N-1)f_{N}(r) \rgr \phi_1(r) ~~=~~ 0, \\
%
\notag
&	\p_r\, \phi_2(r) ~-~ \frac{1}{Nr}\, \lgr f(r) ~-~ f_{N}(r) \rgr \phi_2(r) ~~=~~ 0 ,\\
%
\label{foes}
&	\p_r\, f(r) ~-~ r\, \frac{N g_1^2}{4} \lgr (N-1)\phi_2(r)^2 ~+~ \phi_1(r)^2 ~-~ N\xi \rgr ~~=~~ 0 , \\
%
\notag
&	\p_r\, f_{N}(r)  ~-~  r\, \frac{g_2^2}{2} \lgr \phi_1(r)^2 ~-~ \phi_2(r)^2 \rgr ~~=~~ 0~,
\end{align}
	with the boundary conditions
\begin{align}
%
\label{boundary}
	\phi_1(0) & ~~=~~  0\text,                   & \phi_2(0) & ~~\neq~~ 0\text,  &
	\phi_1(\infty) & ~~=~~ \sqrt{\xi} \text,     & \phi_2(\infty) & ~~=~~ \sqrt{\xi}\text, \\
%
\notag
	f_{N}(0) & ~~=~~ 1\text,                   & f(0) & ~~=~~ 1\text,   &
	f_{N}(\infty) & ~~=~~ 0 \text\,            &  f(\infty) & ~~=~~ 0\text.
\end{align}
	The tension of the $ Z_N $ string \eqref{znstr} is 
\[
	T_1  ~~=~~ 2\pi\xi~.
\]
	
	Besides the position of the center of the string $ x_0 $, a genuine non-Abelian string 
	also possesses collective coordinates in the group space \cfln, which determine the orientation 
	of the string in the group.	
	The solution \eqref{znstr} breaks \cfl down to SU($N-1$) $\times$ U(1).
	Therefore, the space of the orientational coordinates is given by 
\beq
\label{modulispace}
	\frac{\text{SU($N$)}}
            {\text{SU($N-1$)} \times {\rm U(1)}}         ~~\sim~~  \text{CP($N-1$)}\,.
\eeq
	A general non-Abelian string solution can be obtained from Eqs.~\eqref{znstr} by applying
	a \cfl rotation $ U $ to it
\begin{align}
%
\notag
	q ~~=~~ & U\, \lgr \begin{matrix}
			   	\phi_2(r)  & 0  & \ldots & 0 \\
				\ldots  &  \ldots & \ldots & \ldots \\
				0  & \ldots      & \phi_2(r) &  0 \\
				0  & 0           & \ldots  &  \phi_1(r) 
			   \end{matrix}        \rgr     
			U^{-1} \,,
		\qquad \wt{q} ~~=~~ 0\,,
		\\[2mm]
%
\label{nastr}
%
	A_i^\text{SU($N$)} ~~=~~ \frac{1}{N}\, &\, U\, \lgr \begin{matrix}
					          	1  & \ldots & 0 & 0 \\
						  	\ldots & \ldots & \ldots & \ldots \\
							0  & \ldots  & 1  &  0 \\
							0  & 0   & \ldots  &  - (N-1) 
					         \end{matrix} \rgr  \, U^{-1} (\p_i \alpha)\, f_{N}(r)\,,  \\[2mm]
%
\notag
	A_i^{\rm U(1)} ~~=~~ & -\,\frac{1}{N}\, (\p_i \alpha)\, f(r) \cdot \mathlarger{\mathbf{1}}\,,
	\qquad\qquad
			A_0^{\rm U(1)} ~~=~~ A_0^\text{SU($N$)} ~~=~~ 0\,,
	\\
%
\notag
	\aU ~~=~~ & \aN ~~=~~ M^0 ~~=~~ M^a ~~=~~ 0\,,
\end{align}
	where we have switched to the singular gauge in which the quark field does not wind, while the
	gauge field winds around the origin.

	The bosonic string solution \eqref{nastr} does not involve $ \wt{q} $, the adjoint fields $ a $,
	or the $ M $-fields, at the classical level, and thus is independent of the supersymmetry
	breaking parameters.
	In particular, this solution will retain its form when $ \mu_2 $ is taken very large and the 
	adjoints are integrated out.
	Therefore, this solution will still be present in the $M$-model \eqref{mmodel}, 
\begin{align}
%
\notag
	q & ~~=~~ 
		\phi_2 ~+~ n\nbar\, \bigl( \phi_1 ~-~ \phi_2 \bigr) \,,
	\\[2mm]
%%
%\notag
%		& 
%		 ~~=~~  
%			\frac{1}{N}\bigl( \phi_1 ~+~ (N-1)\phi_2 \bigr)
%  			       ~+~ \bigl( \phi_1 ~-~ \phi_2 \bigr)
%			           \lgr n\nbar ~-~ 1/N \rgr \,,
%				   %~~=~~ 
%	\\[2mm]
%
\label{str}
	A_i^\text{SU($N$)} & ~~=~~ \varepsilon_{ij}\, \frac{x^i}{r^2}\, f_{N}(r)
				\lgr n\nbar ~-~ 1/N \rgr,
	\\[2mm]
%
\notag
	A_i^{\rm U(1)} & ~~=~~ \frac{1}{N}\varepsilon_{ij}\, \frac{x^i}{r^2}\, f(r)~, 
	\\[2mm]
%
\notag
	\wt{q} & ~~=~~ M^0 ~~=~~ M^a ~~=~~ 0\,.
\end{align}
	We have introduced here the orientational collective coordinates $ n^l $, which
	parametrize the rotation matrix $ U $ as follows
\beq
\label{n}
	\frac{1}{N}\, U \, \lgr \begin{matrix}
				  1  & \ldots & 0 & 0 \\
				  \ldots & \ldots & \ldots & \ldots \\
				  0 & \ldots & 1 & 0  \\
				  0 & 0 & \ldots & -(N-1) 
				\end{matrix} \rgr
			U^{-1}  
	~~=~~
	-\, n^i\,\ov{n}{}_l  ~~+~~ \frac{1}{N}\cdot{\mathlarger{\mathbf{1}}}{}^i_{~l} ~,
\eeq
	where we deploy matrix notation on the left-hand side.
	The coordinates $ n^l $, $ l = 1, ..., N $ form a complex vector in the fundamental representation
	of SU($N$) and live in the CP($N-1$) space, {\it i.e.}
\beq
\label{unitvec}
		\ov{n}{}_l \cdot n^l ~~=~~ 1,
\eeq
	with one common complex phase of $ n^l $ unfixed ({\it e.g.} one can choose $ n^N $ to be real).

	To obtain the effective sigma model on the world sheet of the string \cite{ABEKY,SYmon,GSY05}, 
	one assumes the moduli $ n^l $ to be slowly-varying functions along the string, $ n^l ~=~ n^l(x^k) $.
	Substituting then the solution \eqref{str} into the kinetic terms of the action \eqref{theory}, one
	arrives at the CP($N-1$) sigma model (see review \cite{SYrev} for details)
\begin{align}
\label{cp}
	S_{\text{CP($N-1$)}}^{1+1} ~~=~~ 2\beta\, \int\, dt\,dz 
					\Bigl\{\, \left|\p n^l\right|^2    
						  ~+~  \left(\nbar \p_k n\right)^2\,
					\Bigr\}\,.
\end{align}
	Here $ \beta $ is the two-dimensional coupling constant which is obtained from an integral of
	the profile functions of the quark and gauge fields over the transverse plane. 
	Using the first-order differential equations \eqref{foes} one can show that the integral
	is in fact a total derivative, and thus, determined by the boundary conditions \eqref{boundary}.
	This yields 
\beq
\label{beta}
	\beta ~~=~~ \frac{2\pi}{g_2^2}\,.
\eeq
	In quantum theory both of the coupling constants entering this equation run, and so one has 
	to specify the scale at which the relation is held.
	It is natural to set the scale of Eq.~\eqref{beta} to the cut-off scale of world-sheet dynamics,
	which is given by the inverse thickness of the string $ g_2 \sqrt{\xi} $.

	Below $ g_2\sqrt{\xi} $ the four-dimensional gauge couplings do not run due to the breaking
	of the gauge symmetry. 
	The two-dimensional coupling starts logarithmic run below the cut-off scale,
\beq
\label{asyfree}
	4 \pi \beta ~~=~~ N\,  \ln \lgr \frac{E}{\Lambda_\text{CP($N-1$)}} \rgr.
\eeq
	By itself the CP($N-1$) theory is asymptotically free \cite{P75}.
%	rendering the world sheet theory asymptotically free \cite{P75}.

%	In the limit of large $ \mu $, the gauge coupling $ g_2 $ is determined by the scale
%	of the \none SQCD \eqref{Lambda}.
%	Using the first coefficient of the $ \beta $ function in \none SQCD, one finds the relation 
%	between the dynamical scales of the two-dimensional and four-dimensional theories \cite{GSYmmodel}
	In the limit of large $ \mu $, the bulk theory becomes \none SQCD with its own scale \eqref{Lambda}.
	Using Eqs.~\eqref{beta} and \eqref{asyfree}, one can find the relation between the scales
	of the world sheet and bulk theories \cite{GSYmmodel}, 
\beq
\label{cpscale}
	\Lambda_\text{CP($N-1$)} ~~=~~ \frac{ \Lambda_{\mc{N}=1}^2 } { g_2\, \sqrt{\xi}}\,,
\eeq
	where the coupling constant is determined by the scale $ \Lambda_{\mc{N}=1} $.

	For a 1/2-BPS string, Eq.~\eqref{cp} gives only half of the world sheet action, {\it i.e.} the bosonic part.
	The fermionic part of the theory is related to the bosonic one by supersymmetry.
	For a string in an \ntwo microscopic theory, world sheet dynamics is given 
	by \ntwot CP($N-1$) sigma model
\begin{align}
%
\notag
\mc{S}_{\rm 1+1}^{\rm (2,2)}  ~~=~~ 2\beta
	\int  d^2x
	\biggl\lgroup\; 
	&
	\left|\p_k n \right|^2  ~+~ \left(\ov{n}\p_k n\right)^2  
	~+~ \ov{\xi}{}_L\, i\p_R\, \xi_L  ~+~ \ov{\xi}{}_R\, i\p_L\,  \xi_R 
	\\[2mm]
%
\label{str_ntwot}
	&\;
	~~-~
	i \left(\nbar\p_R n\right)\, \bxil\xil ~-~ i \left(\nbar\p_Ln\right) \, \bxir\xir 
	\\[2mm]
%
\notag
	&\;
		~~+~
		\bxil \xir \bxir \xil ~-~ \bxir \xir \bxil \xil
	\biggr\rgroup ,
\end{align}
	where $ \xi^i $ is the two-dimensional fermionic superpartner of the orientational moduli $ n^l $.
	The translational sector is completely decoupled from the dynamics of Eq.~\eqref{str_ntwot}.

	When \ntwo supersymmetry is broken, the string internal dynamics is altered, and, as was shown
	in \cite{Edalati}, the world sheet theory is given by \ntwoo \CPCn.
	This theory has one dimensionless coupling $ \gamma $ \cite{SYhet,BSYhet}, which is determined by the
	measure of supersymmetry breaking.
	In two-dimensional theory, this parameter sets the strength of the coupling of the translational
	sector to the orientational one,
\begin{align}
%
\notag
S_{1+1}^{(0,2)} ~~=~~ 2\beta
	\int & d^2 x 
\lgr
	\bzr\, i\p_L\, \zr ~~+~~ \dots 
%\bzl\, i\p_R\, \zl ~~+~~ \left(\p x_0\right)^2 ~+~
\right.
	\\[2mm]
%
\notag
	&
	\;\;
	+~~
	\left|\p n\right|^2 ~~+~~ \left(\ov{n}\p_k n\right)^2 ~~+~~
	\bxir \, i\p_L \, \xir  ~~+~~ \bxil \, i\p_R \, \xil 
	\\[2mm]
%
\label{world02}
	&
	\;\;
	-~~
	i \left(\ov{n}\p_L n\right)\, \bxir \xir ~~-~~ 
	i \left(\ov{n}\p_R n\right)\, \bxil \xil  
	\\[2mm]
%
\notag
	&
	\;\;
	+~~
	\gamma\, (i\p_L\nbar) \xir\zr ~~+~~ \ov{\gamma}\, \bxir (i\p_L n) \bzr ~~+~~
	|\gamma|^2\, \bxil\xil \bzr\zr  
	\\[2mm]
%
\notag
	&
	\;\;
\left.
	+~~ 
	\left( 1 \;-\; |\gamma|^2 \right)\, \bxil\xir \bxir\xil  
	~~-~~ \bxil\xil \bxir\xir
\rgr .
\end{align}
	The ellipses denote the left-handed part of the translational sector, which stays decoupled.

	The world sheet theory \eqref{world02} can be obtained from the gauged formulation of CP($N-1$) \cite{W93}.
	The detail on the derivation of the \CPC action is given in \cite{SYhet,BSYhet}.
	In this formulation, the most natural parameter of the theory $\delta$ arises as 
	a constant at the quadratic deformation of the superpotential
\[
	\mathcal{W}_{1+1} ~~=~~ \frac{1}{2}\,\delta\,\Sigma^2\,,
\]
	where $\Sigma$ is a chiral superfield, a part of the gauge supermultiplet.
The parameter $ \delta $ is related to $ \gamma $ via
\[
	\gamma ~~=~~ \frac { \sqrt{2}\,\delta } { \sqrt{ 1 +  2 |\delta|^2 } }\,.
\]
	The gauged formulation has somewhat more direct physical interpretation than
	the representation \eqref{world02}, as the quantum behavior of the system is more directly
	seen in that picture \cite{SYhet2}. 
	In this sense, $ \delta $ is also a more physical parameter, {\it e.g.}
	$ \delta \to \infty $ supposedly corresponds to a conformal phase of the world sheet theory.
%	The same is true about parameter $ \delta $, in particular, 
%	at $ \delta \to \infty $ the theory supposedly flows to a conformal phase. 

	For a non-Abelian string in \ntwo SQCD broken down to \none by soft mass terms $ \mu_1 $ and $ \mu_2 $
	(see Eq.~\eqref{none}), the relation between $ \gamma $ and $ \mu $ was found to be logarithmic \cite{SYhet,BSYhet}
\beq
	\delta ~~=~~ 
	{\rm const} \cdot \sqrt{\ln\, \frac{g_2^2\mu}{m_W}}
\eeq
	for large $ \mu $.
	The large logarithm arises from the appearance of the Higgs branch in \none SQCD when the adjoint superfields
	become heavy.

	When the $ M $-field is present, the Higgs branch does not appear at large $ \mu $, and the adjoint
	fields are safely integrated out without disrupting the \ntwoo CP($N-1$) theory.
	The two-dimensional parameter $ \gamma $ is then determined by the microscopic parameter $ h $ of the 
	$ M $-model. 
	Below we find the fermionic zero modes in the background of the vortex in the microscopic $M$-model,
	and use them to obtain the relation between these two parameters. 
	
%%%%%%%%%%%%%%%%%%%%%%%%%%%%%%%%%%%%%%%%%%%%%%%%%%%%%%%%%%%%%%%%%%%%%%%%%%%%%%%%%%
%
%	  		        S E C T I O N
%
%%%%%%%%%%%%%%%%%%%%%%%%%%%%%%%%%%%%%%%%%%%%%%%%%%%%%%%%%%%%%%%%%%%%%%%%%%%%%%%%%%
\section{Derivation of the Fermionic Part of the Worldsheet Theory}
\setcounter{equation}{0}

	The fermionic part of the $M$-model \eqref{mmodel} is\footnote{
We denote the fermionic superpartner of the $M$ field as $\vartheta$, in contrast to \cite{GSYmmodel},
where $\zeta$ was used. Here $\zeta$ is reserved for the world sheet supertranslational variable.}
\begin{align}
%
\notag
	\mc{L}_\text{$M$-model}^\text{ferm} & ~~=~~ 
		\frac{2i}{g_2^2}\, \Tr\, \ov{\lambda}{}^\text{SU($N$)} \ov{\slashed{\md}} \lambda^{\text{SU($N$)}}
		~+~ \frac{4i}{g_1^2}\, \ov{\lambda}{}^\text{U(1)} \ov{\slashed{\p}} \lambda^\text{U(1)}
	\\[3mm]
%
\notag
	&~
 		~+~ \Tr\, i\, \ov{\psi \slashed{\md}} \psi  
		~+~ \Tr\, i\, \wt{\psi} \slashed{\md} \ov{\wt{\psi}}
		~+~ \frac{2i}{h}\, \Tr\, \ov{\vartheta\,\slashed{\p}}\,\vartheta
	\\[3mm]
%
\label{fermact}
	&~
		~+~
		i \sqrt{2}\, \Tr 
		\lgr \ov{q}\,\lambda^\text{U(1)}\psi ~-~ \wt{\psi}\,\lambda^\text{U(1)}\ov{\wt{q}}
		 ~+~ \ov{\psi\, \lambda}{}^\text{U(1)} q ~-~ \wt{q}\,\ov{\lambda}{}^\text{U(1)}\ov{\wt{\psi}} \rgr
	\\[3mm]
%
\notag
	&~
		~+~
		i \sqrt{2}\, \Tr
		\lgr \ov{q}\,\lambda^\text{SU($N$)}\psi ~-~ \wt{\psi}\,\lambda^\text{SU($N$)}\ov{\wt{q}}
		 ~+~ \ov{\psi\, \lambda}{}^\text{SU($N$)}q ~-~ \wt{q}\,\ov{\lambda}{}^\text{SU($N$)}\ov{\wt{\psi}} \rgr
	\\[3mm]
%
\notag
	&~
		~+~
		i\,\Tr \lgr
			    \wt{q}\, \psi\,\vartheta ~+~ \wt{\psi}\,q\,\vartheta 
			~+~ \ov{\psi\, \wt{q}\,\vartheta}  ~+~ \ov{q\,\wt{\psi}\,\vartheta} \rgr
	\\[3mm]
%
\notag
	&~
		~+~
		i\,\Tr \lgr
				\wt{\psi}\,\psi\,M ~+~ \ov{\psi\, \wt{\psi}\, M} \rgr ,
\end{align}
where $\lambda^{\alpha}$'s are fermionic \none superpartners of gauge fields, while $\psi^{\alpha}$,
 $\tilde{\psi}_{\dot{\alpha}}$ are matter fermions, $\alpha$, $\dot{\alpha}=1,2$ are spinor indices.
	Here 
\[
	\vartheta^A_B ~~=~~ \frac{1}{2}\,\delta^A_B\, \vartheta^0  ~+~
				(T^a)^A_B\, \vartheta^{a}  ~~\equiv~~
			\frac{1}{2}\,\delta^A_B\, \vartheta^0  ~+~
				(\vartheta^N)^A_B
\]
	is the fermionic superpartner of the $M$ field.
	We need to find the fermionic zero modes in the background of the vortex string \eqref{str}.

	In \cite{GSYmmodel} an index theorem was derived, which shows that this theory possesses
	$ 4 $ $ + $ $ 4 ( N - 1 ) $ zero modes.
	The first four correspond to the fermionic superpartners $ \zeta $ of the bosonic translational moduli
	$ x_0^1 $ and $ x_0^2 $ of the world sheet theory. 
	The other $ 4 ( N - 1 ) $ are the superorientational modes which are associated with the 
	fermionic superpartners $ \xi^l $ of the orientational moduli $ n^l $.

	Since $M$-model possesses half the supersymmetry of the original \ntwo theory, one can utilize
	it in order to find one half of the fermionic zero modes --- the ones that are associated 
	with left-handed fermions of the string world sheet.
	These supertransformations are identical to those of the original \ntwo theory, and the corresponding
	zero modes were calculated in \cite{BSYhet}.

	We have, for the supertranslational modes,
\begin{align}
\label{N2_strans}
%
\notag
\ov{\psi}_{\dot{2}}	& ~~=~~  -\,  2\sqrt{2}\, \frac{x_1 ~+~ i x_2}{N r^2} \,
		\lgr \frac{1}{N} \phi_1 ( f + (N-1) f_N ) ~+~ \frac{N-1}{N} \phi_2 ( f - f_N )  \right.
		\\[2mm]
%
\notag
%
			& \phantom{~~=~~  -\,  2\sqrt{2}}
			~+~ \left( n\nbar ~-~ 1/N \right )
			\Bigl\{ \phi_1 ( f + ( N-1 ) f_N ) ~-~ \phi_2 ( f - f_N) \Bigr\}
		\left. \rgr\, \zeta_L \,,
		\\[2mm]
%
\lambda^{1\ \rm U(1)} 	& ~~=~~ -\, \frac{i g_1^2}{2} \lgr (N-1)\phi_2^2  ~+~ \phi_1^2 ~-~ N\xi \rgr \, \zeta_L \,,
		\\[2mm]
%
\notag
\lambda^{1\ \text{SU($N$)}}	& ~~=~~ -\, {i g_2^2}\, ( n\nbar ~-~ 1/N )\, \lgr \phi_1^2 ~-~ \phi_2^2 \rgr\, \zeta_L\,,
\end{align}
	and 
\begin{align}
\label{N2_sorient}
%
\notag
\overline{\psi}_{\dot{2}Ak} & ~~=~~ \frac{\phi_1^2 ~-~ \phi_2^2}{\phi_2} \cdot n \overline{\xi}_L  \,,
 \\[2mm]
%
\lambda^{1\ \text{SU($N$)}} & ~~=~~ i \sqrt{2}\, \frac{ x^1 ~-~ i\, x^2 }{r^2} 
						  \frac{\phi_1}{\phi_2} f_N \cdot n \overline{\xi}_L \,
\end{align}
	for the superorientational modes. 
	Note that in Eqs.~\eqref{N2_strans} and \eqref{N2_sorient} we listed only the non-vanishing 
	components, which are (by definition) proportional to left-handed fermions. 
	As shown in \cite{GSYmmodel}, the $M$-model has a U(1)$_{\wt R}$ symmetry, under which these components
	have the charge +1. 
	There need to be other, right-handed zero modes which are positively charged 
	under U(1)$_{\wt R}$ as well.
	Although we will not need the explicit expressions for the 
	zero modes \eqref{N2_strans}, \eqref{N2_sorient} in what follows,
	they are helpful in finding a good {\it ansatz} for the right-handed zero modes.
	It is rather obvious that by substituting the modes \eqref{N2_strans} and \eqref{N2_sorient}
	into the kinetic terms of Eq.~\eqref{fermact} one recovers the left-handed kinetic part of the 
	\CPC model \eqref{world02}.

	To obtain the right-handed zero modes one generally needs to solve the Dirac equations.
	We follow the approach of \cite{GSYmmodel} where the parameter $ h $ was tuned to be particularly 
	small (but non-zero),
\beq
\label{smallh}
	0 ~~<~~ h ~~\ll~~ g_2^2\,,
\eeq
	allowing to find the solution analytically.
	We deal with supertranslational and superorientational modes in turn.

	First we make a guess on what fields should participate in the right-handed modes. 
	From the mass-deformed \ntwo case \cite{BSYhet} we know that they would involve
	the fields $ \ov{\wt{\psi}}{}_{\dot 1} $ and $ \lambda^{22} $.
	The latter field is not present in our $M$-model as it was integrated out.
	We infer then that the right set of fermions which constitute the right-handed modes
	are $ \ov{\wt{\psi}}{}_{\dot 1} $, $ \vartheta^0 $ and $ \vartheta^a $, {\it i.e.}
	those which have the U(1)$_{\wt R}$ charges +1 and couple to $ \ov{\wt{\psi}}{}_{\dot 1} $.
	The fermions $ \lambda^{1} $ and $\bar{\psi}_{\dot{2}}$
        decouple from this set completely. They are given by (\ref{N2_strans}), (\ref{N2_sorient}).

\subsection{Supertranslational Zero Modes}

	The Dirac equations for the $ \ov{\wt \psi}{}_{\dot 1} $ and $ \vartheta $ are
\begin{align*}
%
&
	i\, \slashed{\nabla}^{2\dot{1}}\, \ov{\wt\psi}{}_{\dot 1}  
		~+~  i\, q \lgr \frac{1}{2}\, \vartheta_0  ~+~ \vartheta^N \rgr ~~=~~ 0\,, \\
%
&
	\frac{i}{h}\, \ov{\slashed{\p}}{}_{\dot{1}2}\, (\vartheta^N)^2
		~+~ \frac{i}{2}\, {\rm Traceless} \left\{ \ov{q \wt{\psi}}{}_{\dot 1} \right\} ~~=~~ 0\,, \\
%
&
	\frac{i}{h}\, \ov{\slashed{\p}}_{\dot{1}2}\, \vartheta^0 
		~+~ \frac{i}{N}\, {\rm Tr} \lgr \ov{q \wt{\psi}}{}_{\dot 1} \rgr ~~=~~ 0\,.
\end{align*}

For constructing the {\it ansatz} for the solution we extract the trace and traceless components of the
fields, which we mark by superscripts $0$ and $N$, attributing different profile functions to them. 
One has a freedom of placing a factor of $ (x^1 \pm ix^2)/r $ in these components, which gives two possibilities
for the {\it ans\"atze} of the zero modes.
The corresponding world sheet fermions we call $ \zr $ and $ \bzr $,
\begin{align*}
%
	\ov{\wt{\psi}}{}_{\dot 1} ~~=~~ & \frac{1}{2}\, \lgr \cts ~+~ N (n\nbar \;-\; 1/N)\,\ctN \rgr \zr ~+~ \\
%
	& 
		\quad~+~
		\frac{1}{2}\, \frac{x^1 \,-\, ix^2}{r}\, \lgr \pts ~+~ N(n\nbar \;-\; 1/N)\, \ptN \rgr \bzr \\
%
	(\vartheta^0)^2 ~~=~~ &
		\frac{x^1 \,+\, ix^2}{r}\, \rts(r)\cdot \zr 
		~+~ \tts(r)\cdot \bzr \\
%
	(\vartheta^N)^2 ~~=~~ &
		\frac{x^1 \,+\, ix^2}{r}\,\rtN(r) (n\nbar \;-\; 1/N) \cdot \zr 
		~+~ \ttN(r)\cdot (n\nbar \;-\; 1/N) \cdot \bzr \,.
\end{align*}
Here superscript ``tr'' is used to denote the profiles of the supertranslational modes, versus the
superorientational modes to appear later.
Substituting this into the Dirac equations, one obtains the equations 
for the profiles 
$ \chi_{\rm tr}^{0,N} $ and $ \rho_{\rm tr}^{0,N} $
\begin{align*}
%
	&
	\p_r\, \cts ~-~ \frac{1}{Nr}\lgr f\,\cts ~+~ (N-1)\, f_N\, \ctN \rgr \\
	&
	\qquad\qquad
		~+~ i\, \frac{ \phi_1 ~+~ (N-1)\,\phi_2}{N} \rts 
		~+~ 2i\, \frac{N-1}{N^2}\, (\phi_1 ~-~ \phi_2)\, \rtN ~~=~~ 0
\\
%
	&
	\p_r\, \ctN ~-~ \frac{1}{Nr} \lgr f\, \ctN ~+~ f_N\, \cts ~+~ (N-2)\,f_N\,\ctN \rgr \\
	& 
	\qquad\qquad\qquad
		~+~ i\, \frac{\phi_1 ~-~ \phi_2}{N}\, \rts 
		~+~ 2i\, \frac{(N-1)\phi_1 ~+~ \phi_2}{N^2}\, \rtN ~~=~~ 0
\\
%
	&
	-\,\frac{1}{h} \left\{ \p_r ~+~ \frac{1}{r} \right\} \rtN 
		~+~ \frac{i}{4} \lgr (\phi_1 ~+~ (N-1)\,\phi_2)\,\ctN \right. \\
	&
	\qquad\qquad
		\left.
		~+~ (\phi_1~-~\phi_2)\,\cts ~+~ (N-2)\,(\phi_1~-~\phi_2)\,\ctN \rgr ~~=~~ 0
\\
%
	&
	-\,\frac{1}{h} \left\{ \p_r ~+~ \frac{1}{r} \right\} \rts 
		~+~ \frac{i}{2N} \lgr (\phi_1 ~+~ (N-1)\,\phi_2)\,\cts \right. \\
	&
	\qquad\qquad\qquad\qquad\qquad\qquad\quad
		\left.
		~+~ (N-1)\,(\phi_1~-~\phi_2)\,\ctN \rgr ~~=~~0\,,
\end{align*}
	and 
\begin{align}
\label{str_eqn}
%
\notag
	&
	\left\{ \p_r ~+~ \frac{1}{r} \right\} \pts ~-~ \frac{1}{Nr}\lgr f\,\pts ~+~ (N-1)\, f_N\, \ptN \rgr \\
\notag
	&
	\qquad\qquad
		~+~ i\, \frac{ \phi_1 ~+~ (N-1)\,\phi_2}{N} \tts 
		~+~ 2i\, \frac{N-1}{N^2}\, (\phi_1 ~-~ \phi_2)\, \ttN ~~=~~ 0
\\
%
\notag
	&
	\left\{ \p_r ~+~ \frac{1}{r} \right\} \ptN ~-~ \frac{1}{Nr} \lgr f\, \ptN ~+~ f_N\, \pts ~+~ (N-2)\,f_N\,\ptN \rgr \\
	& 
	\qquad\qquad\qquad
		~+~ i\, \frac{\phi_1 ~-~ \phi_2}{N}\, \tts 
		~+~ 2i\, \frac{(N-1)\phi_1 ~+~ \phi_2}{N^2}\, \ttN ~~=~~ 0
\\
%
\notag
	&
	-\,\frac{1}{h}\, \p_r\, \ttN 
		~+~ \frac{i}{4} \lgr (\phi_1 ~+~ (N-1)\,\phi_2)\,\ptN \right. \\
\notag
	&
	\qquad\qquad
		\left.
		~+~ (\phi_1~-~\phi_2)\,\pts ~+~ (N-2)\,(\phi_1~-~\phi_2)\,\ptN \rgr ~~=~~ 0
\\
%
\notag
	&
	-\,\frac{1}{h}\, \p_r\, \tts 
		~+~ \frac{i}{2N} \lgr (\phi_1 ~+~ (N-1)\,\phi_2)\,\pts \right. \\
\notag
	&
	\qquad\qquad\qquad\qquad\qquad\qquad\quad
		\left.
		~+~ (N-1)\,(\phi_1~-~\phi_2)\,\ptN \rgr ~~=~~0
\end{align}
	for the profiles $ \psi_{\rm tr}^{U,N} $ and $ \vartheta_{\rm tr}^{0,N} $.
	Only the second set of equations turns out to yield solutions finite at $ r \to 0 $.
	So we drop $ \chi_{\rm tr} $ and $ \rho_{\rm tr} $, and accept
\begin{align}
%
\notag
	(\vartheta^0)^2 & ~~=~~ \tts(r) \cdot \zr 
\\
%
\label{tr_anz}
	(\vartheta^N)^2 & ~~=~~ \ttN(r)\, (n\nbar ~-~ 1/N) \cdot \zr
\\
%
\notag
	\ov{\wt{\psi}}{}_{\dot 1} & ~~=~~ \frac{1}{2}\, \frac{x^1 \,-\, ix^2}{r} 
						\lgr \pts ~+~ N\,(n\nbar ~-~ 1/N)\, \ptN \rgr \, \zr
\end{align}
	for the zero modes.
	Following \cite{GSYmmodel,BSYhet} we solve Eqs.~\eqref{str_eqn} in the limits of large $r$,
	$ r \gg 1/(g_2\sqrt{\xi}) $ and intermediate $r$, $ r \lesssim 1/(g_2\sqrt{\xi}) $.
 The idea is that if $h$ is small (see (\ref{smallh})) we have  matter fields in our theory
which are much lighter than  gauge bosons, see (\ref{U1mass}), (\ref{SUNmass}). 
Although the light scalars are zero on the string solution, 
the presence of light fermion fields affects the fermionic sector of the theory \cite{SYnone,SYhet}.
In particular, string fermion zero modes have a two-layer structure in the plane
orthogonal to the string axis: a core of the size of the inverse gauge boson mass and long range
"tails" formed by light fermions.
	Completely analogously to the calculations in \cite{GSYmmodel,BSYhet}, in the limit of 
	small $h$, we find in the large-$r$ domain,
\begin{align*}
%
	&& \tts &~~=~~ -\, C\,i\,\frac{m_0^2}{\sqrt{\xi}}\, K_0(m_0 r) 
	& \ttN & ~~=~~ -\, C\,i\,\frac{N}{2}\,\frac{m_0^2}{\sqrt{\xi}}\,K_0(m_0 r) \\
%
	&& \pts &~~=~~ -\, C\,\p_r\,K_0(m_0 r) 
	& \ptN & ~~=~~ -\, C\,\p_r\,K_0(m_0 r)\,,
\end{align*}
	where $ K_0(z) $ is the McDonald function, 
	while at intermediate $ r $ we get
\begin{align*}
%
	\pts & ~~=~~ \ptN ~~=~~ \frac{C}{\sqrt{\xi}}\,\frac{\phi_1}{r}
	\\
%
	\tts & ~~\simeq~~ -\,C\,i\,\frac{m_0^2}{\sqrt{\xi}}\,\ln\frac{m_W}{m_0}
	\\
%
	\ttN & ~~\simeq~~ -\,C\,i\,\frac{N}{2}\,\frac{m_0^2}{\sqrt{\xi}}\,\ln\frac{m_W}{m_0}\,,
\end{align*}
where 
\[
	m_0 ~~\equiv~~ \sqrt{\frac{h}{2}\, \xi}\,.
\]
	The arbitrary constant $ C $ is common for all the profiles, and we can safely put
	it to $ C = 1 $.


\subsection{Superorientational Zero Modes}

Orientational fermion zero modes for $M$ model with gauge group U(2) were calculated in
\cite{GSYmmodel}. Here we generalize these result to the theory with U($N$) gauge group.	
The traceless component of the $ \vartheta $ field now is not involved, and so we
	need to deal only with two Dirac equations
\begin{align}
\label{sor_dirac}
%
&
\notag
	i\, \slashed{\nabla}^{2\dot{1}}\, \ov{\wt\psi}{}_{\dot 1}  
		~+~  i\, q \lgr \frac{1}{2}\, \vartheta_0  ~+~ \vartheta^N \rgr ~~=~~ 0\,, \\
%
&
	\frac{i}{h}\, \ov{\slashed{\p}}{}_{\dot{1}2}\, (\vartheta^N)^2
		~+~ \frac{i}{2}\, {\rm Traceless} \left\{ \ov{q \wt{\psi}}{}_{\dot 1} \right\} ~~=~~ 0\,.
\end{align}
	The form of the zero modes \eqref{N2_sorient} suggests us that the right-handed modes may be proportional
	to $ n\bxir $, or $ \xir\nbar $, which gives two possibilities for the {\it ansatz}.
	One also has the freedom of putting the factor of $ (x^1 \pm ix^2)/r $ to either 
	$ \ov{\wt \psi}{}_{\dot 1} $ or  $ \vartheta^N $.
	Overall, we write the following four {\it ans\"atze}
%\begin{align*}
%
%	&
%	\frac{i}{h}\, \ov{\slashed{\p}}\, \vartheta^a  ~+~
%		i\, \Tr\left( \lgr \ov{\psi\wt{q}} ~+~ \ov{ q\wt{\psi}} \rgr T^a \right) ~~=~~ 0 \\
%%
%	&
%	i\, \slashed{\nabla}\, \ov{\wt{\psi}} ~+~ i\, q\,\vartheta ~~=~~ 0
%\end{align*}
\begin{align*}
%
	(\vartheta^N)^2  & ~~=~~ 2\,\tor(r)\cdot n\bxir ~+~  2\,\eor(r) \cdot \xir \nbar \\
			& \quad
			~+~ 2\,\frac{x^1\,+\,ix^2}{r}\, \kor(r) \cdot n\bxir 
			~+~ 2\,\frac{x^1\,+\,ix^2}{r}\, \sor(r) \cdot \xir\nbar \\
%	
	\ov{\wt{\psi}}{}_{\dot 1} & ~~=~~ 2\, \frac{x^1\,-\,ix^2}{r}\, \por(r)\cdot n\bxir 
				      ~+~ 2\, \frac{x^1\,-\,ix^2}{r}\, \cor(r)\cdot \xir\nbar \\
				& \quad
				          ~+~ 2\, \uor(r)\cdot n\bxir
					  ~+~ 2\, \oor(r)\cdot \xir\nbar\,,
\end{align*}
	where the subscript ``or'' is added to distinguish the profiles from those of the translational modes. 
	Plugging these into the Dirac equations \eqref{sor_dirac}, we have the equations for the profiles
\begin{align}
\label{sor_eqn}
%
\notag
	&
	-\, \p_r\, \tor(r) ~+~ \frac{ih}{2}\, \phi_1(r)\,\por(r) ~~=~~ 0  \\
\notag
	&
	\phantom{-\,}
	\left\{ \p_r ~+~ \frac{1}{r} \right\}\, \por(r)  
			~-~ \frac{1}{Nr}\, \lgr f ~+~ (N-1)\,f_N \rgr \por(r) 
			~+~ i\,\phi_1(r)\,\tor(r) ~~=~~ 0                 \\[4mm]
%
\notag
	& 
	-\, \left\{ \p_r ~+~ \frac{1}{r} \right\}\, \eor(r) 
			~+~ \frac{ih}{2}\, \phi_1(r)\,\cor(r) ~~=~~ 0 \\
	&
	\phantom{-\,}
	\p_r\, \cor(r) ~-~ \frac{1}{Nr}\, \lgr f ~+~ (N-1)\,f_N \rgr \cor(r) 
			~+~ i\, \phi_1(r)\, \eor(r) ~~=~~ 0 \\[4mm]
%
\notag
	&
	-\, \p_r\, \kor(r) ~+~ \frac{ih}{2}\, \phi_2(r)\, \uor(r) ~~=~~ 0 \\
\notag
	&
	\phantom{-\,}
	\left\{ \p_r ~+~ \frac{1}{r} \right\}\, \uor(r) 
			~-~ \frac{1}{Nr} \lgr f ~-~ f_N \rgr \uor(r) 
			~+~ i\,\phi_2(r)\, \kor(r) ~~=~~ 0 \\[4mm]
%
\notag
	&
	-\, \left\{ \p_r ~+~ \frac{1}{r} \right\}\, \sor(r) 
			~+~ \frac{ih}{2}\, \phi_2(r)\,\oor(r) ~~=~~ 0 \\
\notag
	&
	\phantom{-\,}
	\p_r\, \oor(r) ~-~ \frac{1}{Nr}\lgr f ~-~ f_N \rgr \oor(r) 
			~+~ i\,\phi_2(r)\,\sor(r) ~~=~~ 0
\end{align}
From this set of four pairs of equations only the first pair yields non-singular profiles
for the zero modes.
Thus we have, 
\begin{align}
%
\notag
	(\vartheta^N)^2  & ~~=~~ 2\,\tor(r)\cdot n\bxir \\
%
\label{or_anz}
	\ov{\wt{\psi}}{}_{\dot 1} & ~~=~~ 2\, \frac{x^1\,-\,ix^2}{r}\, \por(r)\cdot n\bxir \,.
\end{align}
Again, we solve the equations \eqref{sor_eqn} separately in the domain of large $ r $ and 
intermediate $ r $, assuming $ h $ to be small.
In a fashion similar to \cite{GSYmmodel,BSYhet}, we get for large $ r $, 
\begin{align}
\label{sor_large}
%
\notag
	\tor(r) & ~~=~~ -\, \frac{ih\sqrt{\xi}}{2}\, K_0(m_0 r) \\
%
	\por(r) & ~~=~~ -\, \p_r\, K_0(m_0 r) 
\end{align}
while in the intermediate-$r$ domain the profiles look as
\begin{align}
\label{sor_interm}
%
\notag
	\tor(r) & ~~\simeq~~ -\, \frac{ih\sqrt{\xi}}{2}\, \ln \frac{m_W}{m_0} \\
%
	\por(r) & ~~\simeq~~ \frac{\phi_1}{\sqrt{\xi}\, r}\,.
\end{align}
Mutual normalization of equations \eqref{sor_large} and \eqref{sor_interm} has been taken care of
to ensure agreement between the two domains. 
The overall normalization, similarly to the supertranslational case, is given by a common 
arbitrary constant which we have put to one.

We observe that the right-handed zero modes exhibit long-range $1/r$ behavior similar to that observed
in the \ntwo theory deformed purely by masses \cite{GSYmmodel,SYhet}. 
This is expected, as in the limit $ h \to 0 $ the theory re-acquires the Higgs branch, and the associated
massless modes. 
We have no need, however, of taking this limit, as we took $ h $ small, 
see Eq.~\eqref{smallh}, only for the purpose of making analytical computations simpler.

\subsection{Bifermionic Coupling}

The easiest way to obtain the coupling strength $ \gamma $ of the \CPC theory
is to calculate the strength of the coupling of the supertranslational and superorientational
modes which is induced on the string world sheet
\beq
\label{bif_norm}
	\mc{L}_\text{eff}^{\mc{N}=(0,2)} ~~\supset~~
	2\beta \cdot I_{\zeta\xi}\, ( i\p_L\nbar\,\xir\zr ~+~ \bxir\,i\p_L n\, \bzr )\,,
\eeq
where we separate the factor of $ 2\beta $ for convenience. 
It is natural to assume that $ \gamma $ will be real, since the deformation $h$ is.

To be able to compare this coupling strength to $ \gamma $ in Eq.~\eqref{world02} one has to normalize 
the participating fermions.
We define the normalization integrals $ I_\zeta $ and $ I_\xi $ for the fermions $ \zeta $ and $ \xi $ as
\beq
\label{kin_norm}
	\mc{L}_\text{eff}^{\mc{N}=(0,2)} ~~\supset~~
	2\beta \cdot ( I_\zeta\, \bzr\, i\p_L\,\zr  ~+~ I_\xi\, \bxir\, i\p_L \xir ) \,.
\eeq
Substituting the expressions \eqref{tr_anz} and \eqref{or_anz} for the zero modes into the kinetic terms
of the microscopic theory \eqref{fermact} one obtains
\begin{align*}
%
	I_\zeta & ~~=~~ \frac{N}{2\xi}\, \frac{m_W^2}{4}\, \int\, r\,dr 
		\biggl\lgroup -\, \frac{2}{h}\, 
			  \left\{ \left( \tts \right)^2 
				~+~ 4\,\frac{N-1}{N^2}\, \left( \ttN \right)^2 \right\} \\
		& \qquad\qquad\qquad\qquad\qquad~~
			~+~ \left( \pts \right)^2 ~+~ (N-1)\,\left( \ptN \right)^2  \biggr\rgroup \\
%
	I_\xi & ~~=~~ 2\,g_2^2\, \int\, r\,dr 
		\lgr -\,\frac{2}{h}\bigl( \tor(r) \bigr)^2 ~+~ \bigl( \por(r) \bigr)^2 \rgr.
\end{align*}
From a similar procedure one extracts the expression for the bifermionic coupling
\[
	I_{\zeta\xi} ~~=~~ \frac{g_2^2}{2}\, \int\, r\,dr
		\lgr -\,\frac{4}{h}\,\tor\,\ttN  ~+~ N\, \por\,\ptN  
			~+~ \rho\, \por \left( \pts ~-~ \ptN \right) \rgr.
\]
Substituting here the profile functions of the zero modes, in particular, their $ 1/r $-tails,
we obtain with logarithmic accuracy
\begin{align*}
%
	I_\zeta & ~~=~~ N^2\,\frac{g_2^2}{8}\, \ln \frac{m_W}{m_0}\,, \\
%
	I_\xi   & ~~=~~ 2\,g_2^2\, \ln \frac{m_W}{m_0}\,, \\
%
	I_{\zeta\xi} & ~~=~~ N\, \frac{g_2^2}{2}\,  \ln \frac{m_W}{m_0}\,.
\end{align*}
The logarithms come from the $ 1/m_W \ll r \ll 1/m_0 $ range of the zero mode profiles.

Normalizing the world sheet fermions with the above integrals, one comes to the answer
for the world sheet coupling $ \gamma $ up to a contribution suppressed by the above large logarithms.
Parametrically, the logarithms look as
\[
	\ln \frac{m_W}{m_0}  ~~\simeq~~ 
			\ln \frac{g_2}{\sqrt{h}} ~~=~~ \frac12 \ln \frac{g_2^2}{h}\,,
\]
where $ h $ is small.
Overall the result takes the form
\[
	\gamma ~~=~~ \frac{\sqrt{2}\delta}{\sqrt{1 ~+~ 2|\delta|^2}} ~~=~~ 
		\frac{I_{\zeta\xi}}{\sqrt{I_\zeta\, I_\xi}}  ~~=~~
			1 ~+~ O\lgr \frac{1}{\ln g_2^2/h } \rgr\,,
\]
and so
\beq
\label{deltaresult}
	\delta ~~=~~ \text{const}\cdot \sqrt{\ln g_2^2/h}\,.
\eeq

This result, of course, is very analogous to that for the deformation parameter $ \delta $ 
in the heterotic string scenario \cite{SYhet,BSYhet}. 
The difference, however, is that $ \delta $ does not go to all the way to infinity, and the \CPC model 
gives a reliable description of the string world sheet. 




\section{Confined monopoles}
\setcounter{equation}{0}

As quarks are in the Higgs phase in the bulk theory, the monopoles are confined. It was shown in
\cite{Tong,SYmon,HT2} that when we introduce non-zero FI parameter $\xi$  
in U$(N)$ \ntwo QCD  't Hooft-Polyakov monopoles of SU$(N)$ subgroup become confined on the string --- 
they become string junctions of two elementary non-Abelian strings. Each string of the bulk theory corresponds to a particular vacuum of the world sheet theory. In particular, \ntwot supersymmetric CP$(N-1)$ model on the string 
has $N$ degenerate vacua and kinks interpolating between different vacua. These kinks are interpreted as 
confined monopoles of the bulk theory \cite{Tong,SYmon,HT2}. In the limit of massless quarks
these monopoles become truly non-Abelian. They no longer carry average magnetic flux since
\beq
\langle n^l\rangle =0
\eeq
in the strong coupling limit of CP$(N-1)$ model . Still these monopoles (= kinks) are stabilized by  quantum effects in CP$(N-1)$ model. 
They acquire mass and inverse size of order of $\Lambda_{CP(N-1)}$. They 
are described by fields $n^l$ and form the fundamental representation of the \cfl group \cite{W79}.

Now what happens to these monopoles when we introduce \ntwo supersymmetry breaking parameter
$\mu$ (we assume that $\mu_1\sim\mu_2\equiv\mu$) and take it to infinity flowing to \none QCD? 
Note, that \none QCD has no adjoint fields at all (they decouple), so the monopoles cannot be seen quasiclassically. 
Moreover, no breaking of the gauge group to an Abelian subgroup occur in this theory, 
therefore the monopoles (if exist) should be really non-Abelian.

This question was addressed in \cite{SYhet}, where it was noted that \none QCD  develops a Higgs branch in the 
limit $\mu\to\infty$ 
and therefore the fate of confined monopoles can be followed only up to a
large but still  finite value of $\mu$, see (\ref{critmu}). On the other hand, in the $M$ model
there is no Higgs branch and the presence of confined monopoles was traced all the way to
the limit $\mu\to\infty$ \cite{GSYmmodel}.
In this limit we get a remarkable result: although the adjoint fields 
are eliminated from our theory 
and  monopoles cannot be seen in any semiclassical description,
the  analysis shows
that confined non-Abelian monopoles still exist in the theory (\ref{mmodel}). They are seen
as $CP(N-1)$-model kinks in the effective world-sheet theory on the non-Abelian string.

The only loophole in the above argument is that the fermionic sector of the world sheet theory
was not studied in \cite{GSYmmodel}. In fact, it was not clear, whether the world sheet theory has
$N$ strictly degenerate  vacua and kinks interplating between them (which should be interpreted as
confined  monopoles). Say, if $N$ vacua were split (as it happens in non-supersymmetric case
\cite{GSY05}) monopole and anti-monopole attached to the string would come close together to form a 
meson-like configuration, see review \cite{SYrev} for details. If in the large $\mu$ limit the splittings
were large, bounds inside these mesons could become stronger and individual monopoles would not
be seen. This effect corresponds to kink confinement in two dimensional 
non-supersymmetric CP$(N-1)$ model \cite{W79}.


In this paper we completed the proof of the presence of confined  non-Abelian monopoles
in the $M$ model in the limit $\mu\to\infty$. We have shown above that the world sheet
theory on the non-Abelian string is heterotic \ntwoo supersymmetric CP$(N-1)$ model.
We derived Eq.~(\ref{deltaresult}) which relates the deformation parameter of the world sheet theory
to parameters of the bulk theory in the large $\mu$ limit. In particular, it shows that $\delta$
goes to a constant  at large $\mu$.

Physics of the heterotic \ntwoo supersymmetric CP$(N-1)$ model was studied in the large $N$ approximation in \cite{SYhet2}. In this paper it was shown that supersymmetry is spontaneously broken (see also \cite{Thetdyn}). Still the $Z_{2N}$ discrete symmetry present in the model is
spontaneously broken down to $Z_2$ and the model has $N$ strictly degenerate vacua.
This ensures the presence of kinks, interpolating between these vacua. These kinks are
confined non-Abelian monopoles of the bulk theory.



\section{Conclusions}
\setcounter{equation}{0}

This paper concludes the program started in \cite{GSYmmodel},
namely direct derivation of the world sheet theory  for non-Abelian strings in $M$-model
starting from the bulk theory with \ntwo supersymmetry broken down to \none
by the mass term of the adjoint fields and coupling to the $M$ field.
We showed  that the world sheet
theory on the non-Abelian string is heterotic \ntwoo supersymmetric CP$(N-1)$ model.
To this end we had to explicitly obtain all fermion zero modes
 in the limit of  large $\mu $.
We related the  deformation parameter $\delta$ of the world sheet theory
to parameters of the bulk theory and showed that $\delta$
does not depend on $\mu$ at large $\mu$.

This completes the proof of the presence of non-Abelian monopoles confined on non-Abelian
strings in the $M$ model. Note that at  $\mu\to\infty$ the bulk theory does not have adjoint fields
and monopoles cannot be seen in the quasiclassical approximation. We showed that they are still
present in the theory and seen as  kinks of CP$(N-1)$ model on the string. These kinks
are stabilized by non-perturbative effects (two dimensional instantons) 
when we take the non-Abelian limit in the bulk theory.







\section*{Acknowledgments}

The work of PAB was supported in part by the NSF Grant No. PHY-0554660. PAB is grateful for kind
hospitality to FTPI, University of Minnesota, where a part of this work was done. 
The work of MS was supported in part by DOE grant DE-FG02-94ER408. 
The work of AY was  supported 
by  FTPI, University of Minnesota, 
by RFBR Grant No. 09-02-00457a 
and by Russian State Grant for 
Scientific Schools RSGSS-11242003.2.










%%%%%%%%%%%%%%%%%%%%%%%%%%%%%%%%%%%%%%%%%%%%%%%%%%%%%%%%%%%%%%%%%%%%%%%%%%%%%%%%%%
%%%%%%%%%%%%%%%%%%%%%%%%%%%%%%%%%%%%%%%%%%%%%%%%%%%%%%%%%%%%%%%%%%%%%%%%%%%%%%%%%%
%
%                            B I B L I O G R A P H Y
%
%%%%%%%%%%%%%%%%%%%%%%%%%%%%%%%%%%%%%%%%%%%%%%%%%%%%%%%%%%%%%%%%%%%%%%%%%%%%%%%%%%
%%%%%%%%%%%%%%%%%%%%%%%%%%%%%%%%%%%%%%%%%%%%%%%%%%%%%%%%%%%%%%%%%%%%%%%%%%%%%%%%%%
\small
\begin{thebibliography}{99}
\itemsep -2pt


\bibitem{HT1}
A.~Hanany and D.~Tong,
%``Vortices, instantons and branes,''
JHEP {\bf 0307}, 037 (2003)
[hep-th/0306150].
%%CITATION = HEP-TH 0306150;%%

\bibitem{ABEKY}
R.~Auzzi, S.~Bolognesi, J.~Evslin, K.~Konishi and A.~Yung,
%{\em Non-Abelian superconductors: Vortices and
%confinement in N = 2
%SQCD,}
Nucl.\ Phys.\ B {\bf 673}, 187 (2003)
[hep-th/0307287].
%%CITATION = HEP-TH 0307287;%%

\bibitem{SYmon}
M.~Shifman and A.~Yung,
%``Non-Abelian string junctions as confined monopoles,''
Phys.\ Rev.\ D {\bf 70}, 045004 (2004)
[hep-th/0403149].
%%CITATION = HEP-TH 0403149;%%

\bibitem{Tong}
D.~Tong,
%``Monopoles in the Higgs phase,''
Phys.\ Rev.\ D {\bf 69}, 065003 (2004)
[hep-th/0307302].
%%CITATION = HEP-TH 0307302;%%

\bibitem{HT2}
A.~Hanany and D.~Tong,
%``Vortex strings and four-dimensional gauge dynamics,''
JHEP {\bf 0404}, 066 (2004)
[hep-th/0403158].
%%CITATION = HEP-TH 0403158;%%

\bibitem{SYrev}
M.~Shifman and A.~Yung,
{\sl Supersymmetric Solitons,}
Rev.\ Mod.\ Phys. {\bf 79} 1139 (2007)
[arXiv:hep-th/0703267], also {\sl Cambridge University Press, Cambridge, 2009}.
  %%CITATION = HEP-TH/0703267;%%

\bibitem{Trev}
D.~Tong,
{\em TASI Lectures on Solitons,}
  arXiv:hep-th/0509216.
  %%CITATION = HEP-TH/0509216;%%
 
\bibitem{Jrev}
  M.~Eto, Y.~Isozumi, M.~Nitta, K.~Ohashi and N.~Sakai,
  %``Solitons in the Higgs phase: The moduli matrix approach,''
  J.\ Phys.\ A  {\bf 39}, R315 (2006)
  [arXiv:hep-th/0602170].
  %%CITATION = JPAGB,A39,R315;%%
   
\bibitem{Trev2}
D.~Tong,
{\em Quantum Vortex Strings: A Review,}
  arXiv:0809.5060 [hep-th].
  %%CITATION = ARXIV:0809.5060;%%
  
  \bibitem{Edalati}
  M.~Edalati and D.~Tong,
  %``Heterotic vortex strings,''
  JHEP {\bf 0705}, 005 (2007)
  [arXiv:hep-th/0703045].
  %%CITATION = JHEPA,0705,005;%%

\bibitem{SYhet}
  M.~Shifman and A.~Yung,
  %``Heterotic Flux Tubes in N=2 SQCD with N=1 Preserving Deformations,''
  Phys.\ Rev.\  D {\bf 77}, 125016 (2008)
  [arXiv:0803.0158 [hep-th]].
  %%CITATION = PHRVA,D77,125016;%%
     
\bibitem{SYhet2}
M.~Shifman and A.~Yung,
  %``Large-N Solution of the Heterotic N=(0,2) Two-Dimensional CP(N-1) Model,''
  Phys.\ Rev.\  D {\bf 77}, 125017 (2008)
  [arXiv:0803.0698 [hep-th]].
  %%CITATION = PHRVA,D77,125017;%%
     
     \bibitem{BSYhet}
  P.~A.~Bolokhov, M.~Shifman and A.~Yung,
  %``Description of the Heterotic String Solutions in U(N) SQCD,''
  [arXiv:0901.4603 [hep-th]].
  %%CITATION = ARXIV:0901.4603;%%

  \bibitem{GSYmmodel}
A.~Gorsky, M.~Shifman and A.~Yung,
%``N = 1 supersymmetric quantum chromodynamics: How confined non-Abelian
  %monopoles emerge from quark condensation,'
Phys.\ Rev.\  D {\bf 75}, 065032 (2007)
  [hep-th/0701040].
  %%CITATION = PHRVA,D75,065032;%%

\bibitem{HSZ}
A.~Hanany, M.~J.~Strassler and A.~Zaffaroni,
%``Confinement and strings in M{SSQCD},''
Nucl.\ Phys.\ B {\bf 513}, 87 (1998)
[hep-th/9707244].
%%CITATION = HEP-TH 9707244;%%

\bibitem{VY}
A.~I.~Vainshtein and A.~Yung,
%``Type I superconductivity upon monopole condensation
%in Seiberg--Witten
%theory,''
Nucl.\ Phys.\ B {\bf 614}, 3 (2001)
[hep-th/0012250].
%%CITATION = HEP-TH 0012250;%%

\bibitem{GSY05}
A.~Gorsky, M.~Shifman and A.~Yung,
  %``Non-Abelian Meissner effect in Yang-Mills theories at weak coupling,''
  Phys.\ Rev.\ D {\bf 71}, 045010 (2005)
  [hep-th/0412082].
  %%CITATION = HEP-TH 0412082;%%

\bibitem{P75}
A.~M.~Polyakov,
 %``Interaction Of Goldstone Particles In
 %Two-Dimensions. Applications To
%Ferromagnets And Massive Yang--Mills Fields,''
Phys.\ Lett.\ B {\bf 59}, 79 (1975).
%%CITATION = PHLTA,B59,79;%%

\bibitem{W93}
E.~Witten,
  %``Phases of N = 2 theories in two dimensions,''
  Nucl.\ Phys.\ B {\bf 403}, 159 (1993)
  [hep-th/9301042].
  %%CITATION = HEP-TH 9301042;%%



\bibitem{SYnone}
M.~Shifman and A.~Yung,
%``Non-Abelian flux tubes in N=1 SQCD: supersizing world-sheet supersymmetry,''
Phys.\ Rev.\ D {\bf 72}, 085017 (2005)
[hep-th/0501211].
%%CITATION = HEP-TH 0501211;%%

\bibitem{W79}
E.~Witten,
%``Instantons, The Quark Model, And The 1/N Expansion,''
Nucl.\ Phys.\ B {\bf 149}, 285 (1979).
%%CITATION = NUPHA,B149,285;%%

\bibitem{Thetdyn}
  D.~Tong,
  %``The quantum dynamics of heterotic vortex strings,''
  JHEP {\bf 0709}, 022 (2007)
  [arXiv:hep-th/0703235].
  %%CITATION = JHEPA,0709,022;%%


\end{thebibliography}


\end{document}
